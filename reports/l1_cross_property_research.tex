\documentclass[11pt,a4paper]{article}
\usepackage[margin=2.5cm]{geometry}
\usepackage{amsmath,amssymb,amsthm}
\usepackage{hyperref}
\usepackage{booktabs}
\usepackage{enumitem}

\newtheorem{theorem}{Theorem}
\newtheorem{lemma}[theorem]{Lemma}
\newtheorem{proposition}[theorem]{Proposition}
\newtheorem{corollary}[theorem]{Corollary}
\newtheorem{conjecture}[theorem]{Conjecture}
\theoremstyle{definition}
\newtheorem{definition}[theorem]{Definition}
\newtheorem{remark}[theorem]{Remark}
\newtheorem{question}[theorem]{Question}

\newcommand{\norm}[1]{\lVert #1 \rVert}
\newcommand{\abs}[1]{\lvert #1 \rvert}
\newcommand{\kk}{\mathbb{K}}
\newcommand{\RR}{\mathbb{R}}
\newcommand{\CC}{\mathbb{C}}
\newcommand{\QQ}{\mathbb{Q}}
\newcommand{\Cp}{\mathbb{C}_p}

\title{The Cross Property for Projective Tensor Seminorms:\\
  New Results, Obstructions, and the Finite-Dimensional Reduction}
\author{Research Report --- ProjSeminorm Project}
\date{February 2026}

\begin{document}
\maketitle

\begin{abstract}
  We report on the state of the Cross Property (CP) for the projective
  tensor seminorm over arbitrary nontrivially normed fields~$\kk$.
  We present a new sorry-free Lean~4 proof of CP for spaces with
  $\ell^1$-type norms (no Hahn--Banach, no ultrametric hypothesis).
  We then analyze the expert suggestion of exploiting the
  finite-dimensional reduction, identify the precise obstruction
  (FDNP failure over non-spherically complete fields), and pose
  the key remaining question: does CP hold for all finite-dimensional
  normed spaces over any nontrivially normed field?
\end{abstract}

\tableofcontents

%% ==================================================================
\section{The Cross Property and Known Results}
%% ==================================================================

\subsection{Setup}

Let $\kk$ be a nontrivially normed field, $\{E_i\}_{i\in\iota}$ a finite
family of seminormed $\kk$-spaces.  The \emph{projective seminorm} on
$\bigotimes_{\kk} E_i$ is
\[
  \pi(x) \;=\; \inf\Bigl\{\sum_j \abs{c_j}\prod_i \norm{m^j_i}
    \;:\; x = \sum_j c_j \cdot \bigotimes_i m^j_i\Bigr\}.
\]

\begin{definition}[Cross Property]
  The \emph{Cross Property} (CP) states that for every pure tensor
  $\bigotimes_i m_i$,
  \[
    \pi\bigl(\bigotimes_i m_i\bigr) = \prod_i \norm{m_i}.
  \]
\end{definition}

The upper bound $\pi(\bigotimes m_i)\le \prod\norm{m_i}$ is trivial
(take the one-term representation).
The difficulty is entirely in the \textbf{lower bound}: for every
representation $\bigotimes m_i = \sum_j c_j\bigotimes m^j_i$,
\begin{equation}\label{eq:CP}
  \sum_j \abs{c_j}\prod_i\norm{m^j_i} \;\ge\; \prod_i\norm{m_i}.
\end{equation}

\subsection{Previously Known Cases}

\begin{enumerate}[label=(\alph*)]
\item \textbf{Over $\RR$ or $\CC$} (RCLike.lean):
  Hahn--Banach gives norm-achieving functionals $f_i$ with
  $\norm{f_i}=1$, $\abs{f_i(m_i)}=\norm{m_i}$.
  Evaluating $\bigotimes f_i$ on both sides gives~\eqref{eq:CP}.
\item \textbf{With $h_{\mathrm{bidual}}$} (WithBidual.lean):
  If the canonical embedding $E_i\to E_i^{**}$ is isometric
  for each~$i$, the same duality argument works.
\item \textbf{Ultrametric norms} (SchneiderReduction.lean):
  Over complete non-archimedean fields with ultrametric norms,
  $\varepsilon$-orthogonal bases give near-$\ell^1$ decompositions
  with $(1+\varepsilon)$-loss; taking $\varepsilon\to 0$ closes the proof.
\item \textbf{Collinear representations} (CancellationTrick.lean):
  If all second-factor vectors are collinear ($m^j_i = \alpha_j w_1$
  for $i=2$), bilinearity collapse + the triangle inequality
  (correct direction!) gives~\eqref{eq:CP} with zero duality.
\end{enumerate}

%% ==================================================================
\section{New Result: CP for $\ell^1$-Type Norms}
\label{sec:l1}
%% ==================================================================

\subsection{The $\ell^1$ Decomposition Property}

\begin{definition}
  A basis $b_i$ for a seminormed space $(E_i, \norm{\cdot})$ has the
  \emph{$\ell^1$ decomposition property} if for all $v\in E_i$,
  \[
    \norm{v} = \sum_k \abs{\mathrm{coord}_k(v)}\cdot\norm{b_k}.
  \]
  Equivalently, the unit ball of $\norm{\cdot}$ is a weighted
  cross-polytope in the basis coordinates.
\end{definition}

\subsection{The Proof (L1CrossProperty.lean)}

\begin{theorem}[Sorry-free in Lean~4]
  \label{thm:l1}
  Let $\kk$ be any nontrivially normed field.
  If each factor $E_i$ has a basis with the $\ell^1$ decomposition
  property, then the Cross Property holds:
  $\pi(\bigotimes m_i) = \prod\norm{m_i}$.
\end{theorem}

\begin{proof}[Proof outline]
  Given a representation $\bigotimes m_i = \sum_j c_j\bigotimes m^j_i$:

  \medskip\noindent
  \textbf{Step 1} (Coordinate extraction via \texttt{dualDistrib}).
  For each index tuple $\sigma = (\sigma_i)_{i\in\iota}$,
  applying the multilinear functional
  $\bigotimes_i \mathrm{coord}_{\sigma(i)}$ to both sides:
  \[
    \prod_i \mathrm{coord}_{\sigma(i)}(m_i)
    \;=\; \sum_j c_j \prod_i \mathrm{coord}_{\sigma(i)}(m^j_i).
  \]

  \textbf{Step 2} (Per-tuple inequality).
  Triangle inequality + norm multiplicativity:
  \[
    \prod_i \bigl(\abs{\mathrm{coord}_{\sigma(i)}(m_i)}\cdot
      \norm{b_{\sigma(i)}}\bigr)
    \;\le\;
    \sum_j \abs{c_j}\prod_i \bigl(\abs{\mathrm{coord}_{\sigma(i)}(m^j_i)}
      \cdot\norm{b_{\sigma(i)}}\bigr).
  \]

  \textbf{Step 3} (Sum over all tuples + product-sum swap).
  Using \texttt{Fintype.prod\_sum}
  ($\prod_i\sum_k f(i,k) = \sum_\sigma\prod_i f(i,\sigma(i))$):
  \begin{align*}
    \prod_i\norm{m_i}
    &\;=\; \prod_i\sum_k \abs{\mathrm{coord}_k(m_i)}\cdot\norm{b_k}
      &&\text{($\ell^1$ hypothesis, LHS)}\\
    &\;=\; \sum_\sigma \prod_i
      \abs{\mathrm{coord}_{\sigma(i)}(m_i)}\cdot\norm{b_{\sigma(i)}}
      &&\text{(\texttt{Fintype.prod\_sum})}\\
    &\;\le\; \sum_\sigma\sum_j\abs{c_j}\prod_i(\cdots)
      &&\text{(Step 2, \texttt{Finset.sum\_le\_sum})}\\
    &\;=\; \sum_j\abs{c_j}\sum_\sigma\prod_i(\cdots)
      &&\text{(\texttt{Finset.sum\_comm})}\\
    &\;=\; \sum_j\abs{c_j}\prod_i\sum_k
      \abs{\mathrm{coord}_k(m^j_i)}\cdot\norm{b_k}
      &&\text{(\texttt{Fintype.prod\_sum}, reverse)}\\
    &\;=\; \sum_j\abs{c_j}\prod_i\norm{m^j_i}.
      &&\text{($\ell^1$ hypothesis, RHS)}\qedhere
  \end{align*}
\end{proof}

\subsection{Where the $\ell^1$ Hypothesis Is Used}

The $\ell^1$ property is used in \textbf{both directions}:
\begin{itemize}
\item \textbf{LHS} (first line):
  $\norm{m_i} = \sum_k\abs{\mathrm{coord}_k(m_i)}\cdot\norm{b_k}$
  --- converts the norm to a sum-over-coordinates.
\item \textbf{RHS} (last line):
  $\sum_k\abs{\mathrm{coord}_k(m^j_i)}\cdot\norm{b_k} = \norm{m^j_i}$
  --- converts the sum-over-coordinates back to a norm.
\end{itemize}

Without the $\ell^1$ property, only the triangle inequality holds:
$\norm{v} \le \sum_k\abs{\mathrm{coord}_k(v)}\cdot\norm{b_k}$.
This gives the \emph{wrong direction} for the LHS (produces $\ge$ where
we need $=$) and the \emph{correct direction but too large} for the RHS.

%% ==================================================================
\section{The Expert's Suggestion: Finite-Dimensional Reduction}
\label{sec:fd}
%% ==================================================================

\subsection{The Observation}

Every representation $\bigotimes m_i = \sum_{j=1}^n c_j\bigotimes m^j_i$
involves finitely many vectors.  For each factor~$i$, let
\[
  V_i \;=\; \mathrm{span}_\kk\bigl(\{m_i\}\cup\{m^j_i:j=1,\ldots,n\}\bigr)
  \;\subset\; E_i.
\]
Then $\dim V_i < \infty$, the representation lives entirely in
$\bigotimes V_i$, and the cost $\sum_j\abs{c_j}\prod_i\norm{m^j_i}$
is computed using the restricted norms on~$V_i$.

\subsection{The Proposed Argument}

\begin{enumerate}
\item For each representation, restrict to $V_i$ (finite-dimensional).
\item In $V_i$, prove CP (by some finite-dimensional argument).
\item Since every representation has cost $\ge\prod\norm{m_i}$,
  the infimum (projective seminorm) is also $\ge\prod\norm{m_i}$.
\end{enumerate}

Step~3 is logically correct.  The entire question reduces to:

\begin{question}[The Key Question]
  \label{q:fdcp}
  Does the Cross Property hold for all finite-dimensional normed spaces
  over any nontrivially normed field?
\end{question}

\subsection{What Works in Finite Dimensions}

\begin{itemize}
\item \textbf{Over $\RR$ or $\CC$}: Hahn--Banach $\Rightarrow$
  isometric bidual embedding $\Rightarrow$ CP. \checkmark
\item \textbf{Over spherically complete NA fields} ($\QQ_p$, etc.):
  Ingleton's theorem gives Hahn--Banach $\Rightarrow$ isometric bidual
  $\Rightarrow$ CP. \checkmark
\item \textbf{Over non-spherically complete NA fields} ($\Cp$, etc.):
  The bidual embedding is \textbf{not} isometric, even in
  dimension~2!
\end{itemize}

\subsection{The FDNP Obstruction}
\label{sec:fdnp}

\begin{definition}[Finite-Dimensional Norming Property]
  A normed space~$V$ over~$\kk$ has the \emph{FDNP} if
  $\sup_{\norm{f}\le 1}\abs{f(v)} = \norm{v}$ for all $v\in V$,
  i.e., the canonical embedding $V\hookrightarrow V^{**}$ is isometric.
\end{definition}

\begin{proposition}[Confirmed in session~16 of this project]
  There exists a $2$-dimensional normed space over $\Cp$ where
  FDNP fails: $\sup_{\norm{f}\le 1}\abs{f(v)} < \norm{v}$ for
  some unit vector~$v$.
\end{proposition}

This means the $h_{\mathrm{bidual}}$-based proof of CP fails in finite
dimensions over~$\Cp$.

\begin{remark}[Hahn--Banach Extension vs.\ Norming]
  Over non-spherically complete fields, spaces of ``countable type''
  (including all finite-dimensional spaces) still enjoy the
  Hahn--Banach \emph{extension} property: bounded functionals on
  subspaces extend to the whole space with the same norm.
  However, this does \textbf{not} imply that norm-\emph{achieving}
  functionals exist.  The extension property preserves the norm of the
  \emph{functional}, not the norm of the \emph{vector}.
\end{remark}

\subsection{CP $\ne$ FDNP}

Crucially, the Cross Property is \textbf{strictly weaker} than FDNP:

\begin{itemize}
\item FDNP requires a single norm-achieving functional for each vector.
\item CP requires that the \emph{infimum over all representations}
  equals the product of norms --- a global property of the tensor product.
\end{itemize}

In the $2$-dimensional $\Cp$-space where FDNP fails, exhaustive
numerical search found \textbf{no CP counterexample} (see project
session~16, CROSS\_PROPERTY~\S2.2).  This suggests CP may hold
even where FDNP fails.

%% ==================================================================
\section{Analysis of Proposed Approaches}
\label{sec:approaches}
%% ==================================================================

\subsection{Banach--Mazur Distance to $\ell^1$}

Every $n$-dimensional normed space has Banach--Mazur distance
$d_{BM}(X,\ell^1_n)\le n$ to $\ell^1_n$.  If we could embed each
factor into an $\ell^1$ space and apply Theorem~\ref{thm:l1},
the distortion would be $\prod_i d_{BM}(V_i,\ell^1_{n_i})$, which
is \textbf{multiplicative across factors} and dimension-dependent.
This is fatal --- it gives a factor that can be as bad as
$\prod_i n_i$ and does not tend to~1.

\textbf{Verdict:} Not viable.

\subsection{Representation-Dependent Basis Choice}

Since the basis can depend on the specific representation, one might
hope to find a ``good'' basis for the specific vectors involved.
However:

\begin{itemize}
\item For a \emph{single} vector $v$, one can always choose a basis
  with $v = \alpha\cdot e_1$, giving $\norm{v}=\abs{\alpha}\norm{e_1}$
  (the $\ell^1$ property holds trivially for 1 vector).
\item For a \emph{set} of vectors $\{m_i, m^j_i\}$, no basis
  generally makes the $\ell^1$ property hold simultaneously for all
  of them.
\end{itemize}

\textbf{Verdict:} Not viable in its naive form.

\subsection{Permanent / Capacity Bounds (Gurvits)}

The product-sum swap $\prod_i\sum_k f(i,k)=\sum_\sigma\prod_i f(i,\sigma(i))$
is a permanent of a rectangular matrix.  Gurvits' capacity technique
proves $\mathrm{perm}(A)\ge n!/n^n$ for doubly stochastic~$A$
via H-stable polynomials.

\textbf{Obstruction:} The capacity technique requires real stability
(roots avoid the upper half-plane), which is tied to~$\RR/\CC$.
No non-archimedean analogue is known.

\textbf{Verdict:} Interesting connection but not directly applicable
over general fields.

\subsection{Tropical / Valuative (Already Done)}

The ultrametric (Schneider) approach is essentially the tropical
version of our argument: the max-norm replaces the sum-norm,
and $\varepsilon$-orthogonal bases give near-$\ell^1$ structure.
This is already formalized in SchneiderReduction.lean.

\textbf{Verdict:} Complete for ultrametric norms; does not extend
to archimedean.

%% ==================================================================
\section{The Remaining Question and Strategy}
\label{sec:strategy}
%% ==================================================================

\subsection{The Reduction}

By the finite-dimensional reduction of Section~\ref{sec:fd},
the unconditional CP reduces to:

\begin{conjecture}[Finite-Dimensional Cross Property]
  \label{conj:fdcp}
  For any nontrivially normed field~$\kk$, any finite family of
  finite-dimensional normed $\kk$-spaces $\{V_i\}$, and any
  $m_i\in V_i$,
  \[
    \pi_{V_1\otimes\cdots\otimes V_n}\bigl(\bigotimes m_i\bigr)
    = \prod\norm{m_i}.
  \]
\end{conjecture}

If Conjecture~\ref{conj:fdcp} is true, the unconditional CP follows
immediately: every representation in the full (possibly
infinite-dimensional) tensor product restricts to a finite-dimensional
subproduct where the conjecture applies.

\subsection{What Would Prove the Conjecture}

A proof of Conjecture~\ref{conj:fdcp} must avoid $h_{\mathrm{bidual}}$
(which fails over~$\Cp$).  Possible strategies:

\begin{enumerate}
\item \textbf{Direct metric-algebraic argument} in finite dimensions.
  The cancellation trick works for rank-1; can it be extended?
\item \textbf{Weaker-than-$\ell^1$ decomposition property}
  that holds for all FD norms and still suffices for the
  product-sum-swap argument.
\item \textbf{Induction on dimension}: CP holds in dimension~1
  (trivially) and dimension~2 (empirically over~$\Cp$).
  An inductive step reducing $\dim V_i = d$ to $d-1$ would suffice.
\item \textbf{Approximation by $\ell^1$ norms}: If every FD norm
  can be approximated (in a suitable sense) by $\ell^1$ norms,
  and if CP is continuous under this approximation, then
  Theorem~\ref{thm:l1} would give the result.
\end{enumerate}

\subsection{What We Know About Approximation}

For the approximation strategy, we would need: for every $\varepsilon>0$
and FD normed space $(V,\norm{\cdot})$, there exists a norm
$\norm{\cdot}_1$ with the $\ell^1$ property such that
$(1-\varepsilon)\norm{v}_1 \le \norm{v} \le \norm{v}_1$ for all~$v$.

The first inequality says the $\ell^1$ ball contains $(1-\varepsilon)$
times the norm ball.  The second says the norm ball is inside the
$\ell^1$ ball.  The second always holds (triangle inequality).
The first requires the norm ball to be approximated from inside by
a cross-polytope, which is possible with $\varepsilon$
depending on~$\dim V$ (John's theorem gives
$\varepsilon \ge 1-1/\sqrt{\dim V}$, too large).

\textbf{This approach fails for the same reason as Banach--Mazur:}
the approximation quality degrades with dimension.

%% ==================================================================
\section{Conclusions}
\label{sec:conclusions}
%% ==================================================================

\begin{center}
\begin{tabular}{lll}
\toprule
\textbf{Approach} & \textbf{Status} & \textbf{Obstruction} \\
\midrule
$h_{\mathrm{bidual}}$ (duality) & Conditional & FDNP fails over $\Cp$ \\
Schneider (ultrametric) & Complete & Only ultrametric norms \\
$\ell^1$ norms (this work) & \textbf{Complete} & Only $\ell^1$ norms \\
Banach--Mazur to $\ell^1$ & Failed & Multiplicative distortion \\
Representation-dependent basis & Failed & Cannot force $\ell^1$ for multiple vectors \\
Permanent / capacity & Blocked & Requires real stability \\
FD reduction & \textbf{Reduces to Conj.~\ref{conj:fdcp}} & Need FD-CP without $h_{\mathrm{bidual}}$ \\
\bottomrule
\end{tabular}
\end{center}

\bigskip

The key remaining question is Conjecture~\ref{conj:fdcp}: does CP hold
for all finite-dimensional normed spaces over any nontrivially normed
field?  If yes, the unconditional CP follows.  If no, a
finite-dimensional counterexample over~$\Cp$ would settle the problem
negatively.

The $\ell^1$ cross property (Theorem~\ref{thm:l1}) provides a new
sorry-free proof technique that works over \emph{any} nontrivially
normed field, using only the triangle inequality and the
$\texttt{Fintype.prod\_sum}$ identity.  This is the first CP proof
that requires neither duality nor the ultrametric property.

%% ==================================================================
\section*{Formalization Status}
%% ==================================================================

All results in this report are formalized in Lean~4 (mathlib v4.27.0):
\begin{itemize}
\item 10 source files, $\sim$1060 LOC, 0 sorries
\item \texttt{L1CrossProperty.lean}: Theorem~\ref{thm:l1} (140 LOC)
\item Axioms used: \texttt{propext}, \texttt{Classical.choice},
  \texttt{Quot.sound} (standard Lean foundations only)
\end{itemize}

\end{document}
